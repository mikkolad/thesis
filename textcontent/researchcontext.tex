\section*{Preface}
In this thesis the primary question we seek to understand better is "how do stars move in the Galaxy?". To shed light on this I have examined radial migration in simulated Milky Way-like galaxies to constrain its dependencies. Furthermore, I have estimated the full 3D velocity distributions of different samples of local stars using the largest available astrometric catalogues from the \textit{Gaia} mission. The velocity distributions have allowed to demonstrate unknown properties of the white dwarf population as well as identify new velocity substructure which can tentatively be linked to accretion events.\newline
\newline
This thesis summarises the work which has been published in two papers and is to be submitted in a third. Here follows a brief description of each paper to give the reader an overview of the thesis.

\begin{enumerate}
    \item \textit{Paper I: Radial migration and vertical action in N-body simulations} - My first paper seeks to determine how effectively stars are radially migrated as a function of how vertically extended their orbit is and how gravitationally dominant their stellar disc is. This was an interesting question since the current understanding of radial migration, following works by \cite{solway:12, vera-ciro:14, vera-ciro:16b}, especially the latter two, had firmly argued for what they call \textit{provenance bias}; radial migration primarily affects stars with small vertical excursions. We investigated whether this \textit{provenance bias} exists regardless of the strength of the spiral arms that cause the radial migration. This had also been investigated in the third of the cited papers, which found that it indeed was. To do this we set up our own simulations with a pure $N$-body bulge, halo, and disc. We used 11 different set-ups where the halo mass is varied. This allowed the force ratio of halo to disc, $F_\mathrm{h}/F_\mathrm{d}$ to vary between 0.5 to 3.2 at a Solar orbit. In other words, we went from strongly disc-dominated systems to halo-dominated systems. We could then quantify the radial migration by comparing the final and initial angular momentum $\Delta L_z = L_{z,f} - L_{z_i}$, and compare it to the vertical action. For a region of the disc, the migration is better described by the spread of $\Delta L_z$, described by its standard deviation $\sigma_{\Delta L_z}$. We looked at the slope of the function that relates the migration efficiency parameter to the vertical action, $J_z$, at various radii of the disc in all of our simulations. This let us show that the as the disc becomes more dominant, the slope flattens, and the provenance bias disappears across the whole disc. This was not the case when migration efficiency was compared to radial action, which implied that there is a difference in the response of migration to increased actions in the various directions. We also recreated the simulations used by \cite{vera-ciro:16b} and were able to reproduce our results there as well.
    
    \item \textit{Paper II: The velocity distribution of white dwarfs in Gaia EDR3} - In the second paper the goal is to determine accurate velocity distributions for white dwarfs (WDs) without having to rely on measured line-of-sight velocities. The data releases of \textit{Gaia} have revolutionised the field of astrometry and has produced a truly great catalogue of stellar motions and positions. Already in EDR3 the catalogue contained the full astrometric solution for up to ${\sim}$1.5 billion sources which completely dwarfs the sample when limited to measured line-of-sight velocities which is about ${\sim}$7.2 million or about 0.5\%. It is typically difficult to get radial velocities for WDs since their spectra has few and broad lines, which means that working with pure proper motions is preferred in order to access a large sample of them. Doing this allows us to use a large sample of 129 675 WDs within 500 pc. Inferring the velocity distribution from proper motions and positions was done for \textit{Hipparcos} data by \cite{dehnen:98a} using a penalised maximum likelihood estimate. The method requires a sample that can be assumed to have proper motions uncorrelated to the on-sky positions. In this case the proper motions in one part of the sky compensates for the missing line-of-sight velocities of another. We applied this method and the related method of \cite{dehnen:98b} to determine the velocity distributions and velocity dispersions for the WD population. Gaia DR2 had already showed that the WDs have a bifurcation in the colour-magnitude diagram, so the difference between the two sequences was something we chose to investigate. We were able to show that the bluer sequence has lower velocity dispersion, across all magnitudes where the bifurcation exists, than the redder sequence. This means that they comprise two separate kinematic populations. Furthermore, we determined the statistical independence of the two populations with a KS-test. The current best explanation for the bifurcation is atmospheric composition, which would not have a bearing on the kinematics. Our results therefore provide support for alternative explanations. 
        
    \item \textit{Paper III: New stellar halo substructures from Gaia DR3 proper motions} - The third paper expands the use of the penalised likelihood estimate from Paper II. The benefits of using proper motion catalogues warranted further exploitation and so we decided to apply the method towards two new samples: the Solar neighbourhood and the local stellar halo. For the Solar neighbourhood this results in 1 171 846 stars within 200 pc with 10\% parallax uncertainty. This sample gives a very accurate view of what the velocity distribution looks like at `face-value', with all of the most well-known moving groups featuring in the distribution. This also demonstrates that in order to find additional velocity substructure in the Solar neighbourhood special methods must be applied and in our case, we made use of conditional probabilities of one velocity component of the distribution upon the other. This effectively normalises the distribution along either the rows or the columns of the 2D map and reveals further structure in high-velocity regions. Selecting the local (within 3 kpc distance) stellar halo with a velocity cut of $v_\mathrm{T} > 200$ km s$^{-1}$ provides us with 456 273 stars. This also reveals the double main sequences which have been previously identified in the colour-magnitude diagram in \textit{Gaia} DR2 \cite{dr2:hr}. The sequences are typically associated with redder `in-situ' halo and a blue `accreted' halo which matches well a lower metallicity isochrone. This second sequence has been shown to have significant amounts of accreted substructure (see e.g., \citealt{koppelman:19} and links therein). We create a sample for each sequence with 239 115 and 194 507 stars in left and right sequences respectively. We focus on the accreted population and identify the majority of substructures available in literature. Using the conditional probabilities, we can identify two new features the the accreted halo which have no match in literature: \textit{MMH-1} at $(v_r, v_\phi, v_\theta) \approx (\pm220, 20, 300)$ km s$^{-1}$ and \textit{MMH-2} at $(v_r, v_\phi, v_\theta) \approx (-20, 180, -100)$ km s$^{-1}$.
\end{enumerate}

Since the first of the articles is on dynamics in $N$-body simulations while the second and third are on statistical analysis of astrometric data, the thesis is divided into chapters reflecting this. Chapter 1 gives a structural overview of the Milky Way and explains radial migration. This follows naturally into Chapter 2 which summarises the first article. Chapter 3 gives historical context for the field of astrometry and background for the second and third article's method. Finally, Chapters 4 and 5 summarise the second and third articles respectively. 