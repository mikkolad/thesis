\section{Outlook}\label{sect:outlook}

%The nuclear star cluster of the Milky Way is still poorly investigated with high resolution spectro\-scopy, which is needed for an accurate and precise abundance determination. Expanding our stellar sample to 100 stars or more would allow us to determine metallicities and abundances to a much higher degree, allowing us to confirm or reject the result suggested in Paper V. If the nuclear star cluster indeed is different from other stellar populations found elsewhere in the Galaxy, it is important to collect a good high resolution sample to map out the differences.
%
%It is possible to increase the resolution beyond the 23000 we have used, since for example NICRSPEC on the Keck telescopes has recently been upgraded to 38000, and other spectrometers exist, like IGRINS with a resolution of about 56000. IGRINS is also able to capture the H band ($\sim$\,1.5\,micron) in addition to the K band ($\sim$\,2\,micron), opening up an abundance analysis of many more chemical speciecs, though the higher extinction in the H band is a problem with current telescopes. The higher resolution will allow for higher fidelity in line detection and blend analysis.
%
%We are also interested in the exploration of the nuclear stellar disk. It is about 100 times more massive than the nuclear star cluster, and could be a more important component to study than the nuclear star cluster. We are planning on embarking on such a project with our collaborator R. M. Rich.
%
%In our current collected data set we have a few stars still under investigation that we are curious about (Thorsbro et al., in prep.). These stars are different from the stars included in our published work in that they are most probably young stars found in the nuclear star cluster. Their presence opens up the possibility of mapping the more recent star formation. Compared to earlier studies of these stars it is possible that we can advance our understanding of these stars by having more accurate and precise metallicity and abundance determinations from high resolution spectroscopy.
%
%We have also noticed a particularly high light element abundance in the stars we have observed, most likely nitrogen (Thorsbro et al., in prep.). It will be interesting to study this in further detail and determine if there is an actual difference between stars in the Galactic centre and stars further out in this respect. If this turns out to be an additional difference it will be another interesting observation that needs to be explained by theories.
%
%Finally, with the new large telescopes, 30\,m or more in diameter, new observation targets becomes available to us. Apart from seeing things further away, like stars in the Andromeda galaxy, M31, they can also be used for observing shorter wavelength regimes in dust-extincted regions compared to what we are able to do today. Being able to observe the nuclear star cluster in the H band could open up the abundance analysis of elements that do not have spectral lines in the K band.
%
%The future is bright.
