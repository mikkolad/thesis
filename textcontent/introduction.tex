\section{Introduction}

%In astrophysics, one of the current and active research topics is understanding the formation and evolution of galaxies \citep{blandhawthorn:16}. Our own galaxy, the Milky Way Galaxy, plays a crucial role, not only because it is the galaxy that we live in, but also because it is the galaxy that we can study in the most detail. Assuming the mediocrity principle, i.e.\ that we are not special, we can use the understanding of our own galaxy to understand galaxies at large.
%
%Our Galaxy is often described as a barred spiral galaxy consisting of number of components, often classified as the halo, the thin and thick disks, the bulge and the Galactic centre \citep{blandhawthorn:16}. Zooming in on the Galactic centre we find that the innermost parts can be described as consisting of a nuclear star cluster hosting a super-massive black hole, with the cluster itself being surrounded by a nuclear stellar disk.
%
%The innermost environment of a galaxy is important for several reasons. Firstly, there are empirical relations between the nuclear star cluster and the entirety of the galaxy it is hosted in. For example we find an empirical relation between the mass of the nuclear star cluster and the mass of the host galaxy. Secondly, almost all galaxies that we observe are found to contain a nuclear star cluster. Thirdly, nuclear star clusters are extremely dense and bright star environments visible from a long way away \citep[see e.g.\ review by][]{neumayer:20}.
%
%In our own Galaxy many of the individual stars of the inner Galactic centre can be resolved and observed separately. Through the technique of spectroscopy the light emitted from these stars can be examined in extreme detail giving us insight into information about the stars, most notably the chemical composition of the stars \citep[see e.g.][]{spectrophysics:1999,stellaratmospheres:2014}.
%
%All chemical species beyond the very lightest species are formed by various different fusion processes in the stars, or in some cases during the merger processes of stars \citep{kobayashi:20}. Thus, the observation of different chemical species are a consequence of, and hence a witness to, the existence of these fusion processes. Which fusion process happens is dependent on several factors, most notably the mass of the star and whether or not the star is part of a binary star system, described in more detail in stellar evolution theories \citep[see e.g.][]{prialnik:00}. Chaining together the lives of many stars accumulates a production of chemical species that eventually combines into a mixture of chemical species that we can observe today. The chemical species mixtures are thus fingerprints of the history of the given environments the mixtures are observed in.
%
%In other words, by observing the chemical composition of stars, we are able to set constraints on, or provide clues to, how the environments that stars reside in, have formed and evolved.
%
%In Section~\ref{sect:galaxyevolution} galaxy formation and evolution theories and how they relate to the work in this thesis are described briefly and in Section~\ref{sect:chemicalevolution} an overview of basic chemical evolution is given. Following in Section~\ref{sect:observations} is an account of the observations that have provided the data. Finally, in Section~\ref{sect:spectroscopy} the methodology of spectroscopy, which is the foundation of the work done in this thesis, is described in more detail. In the second part of this thesis the author's contributions to the scientific articles are described and the articles themselves are included.