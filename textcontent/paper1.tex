\chapter{Paper I}\label{chap:paper1}
\section{Introduction}\label{sec:p1-intro}
In my first paper the aim is to establish the relationship between the efficiency of radial migration by cold torquing and the vertical action of stars. We seek to answer the question: do stars migrate equally, or less efficiently, with large vertically extended orbits? In order to study radial migration effectively, we make use of numerical or `$N$-body' simulations since it is a process which occurs over hundreds of millions of years. In addition to that, using isolated disc galaxy simulations lets us have a great deal of control over the parameter space. 

The paper particularly focuses on determining the efficiency of radial migration as a function of the kinematics of their orbits. This is not a topic that has gone entirely without study of course. The efficiency of radial migration as a function of radial velocity dispersion is investigated in \cite{solway:12}, \cite{vera-ciro:14}, and \cite{daniel:18} who all agree that migration is reduced with increased radial motion. A similar trend can be seen when investigating vertical motion or vertical scale height. This is studied by several articles including \cite{solway:12, vera-ciro:14, halle:2015, vera-ciro:16b} and the conclusion is the same: radial migration is reduced with increased vertical excursion measured either through scale height or velocity dispersion. This is dubbed the \textit{provenance bias} by \cite{vera-ciro:14}. These results can be understood quite simply by considering the interaction between a star and a spiral arm. Cold, circular orbits will spend more time near the corotation resonance of the spiral and be more prone to migration. In \cite{vera-ciro:16b} they create three different simulations of live discs in static dark matter halo potentials. These go from a lighter disc to a heavier disc, which will result in fewer, stronger, spirals \citep{donghia:15}. They claim that the provenance bias is present regardless of morphology. However, a sufficiently strong spiral arm from a massive stellar disc, could perhaps be strong enough to migrate stars that reach larger vertical excursions as well. 

In the paper we seek to answer how this provenance bias is affected by spiral morphology and disc dominance. To do this, we generate a large number of $N$-body simulations where the ratio of the halo to disc strength is varied to produce discs with differently sized spiral arms. We investigate the radial migration that occurs in these simulations and quantify it as a function of the disc dominance, specifically the provenance bias of the migration. We determine that the slope of radial migration efficiency as a function of vertical action is strong for weaker discs (i.e., a provenance bias exists) and flattens for stronger discs, supporting the idea that strong spirals can migrate vertically extended stars. For radial action we find that there is a provenance bias regardless. 

\section{Setting up simulations}\label{sec:p1-simulations}
All of our simulations are setup and run using packages available as part of the \textsc{nemo}\footnote{\url{https://teuben.github.io/nemo/}}\citep{teuben:95} toolbox. We start with a pure $N$-body isolated galaxy system designed to be similar to the Milky Way by using values from \cite{mcmillan:17}. The full specifics of each component and the chosen parameters are found in paper I.

We follow the procedure of \cite{mcmillan:07} to generate a disc with \textsc{mkwd99disc}. This disc has a standard shape with a density profile that decreases exponentially in radius and has a $\mathrm{sech}^2$ vertical profile:
\begin{equation}
    \rho_\mathrm{disc}(R, z) = \frac{1}{2z_\mathrm{d}} \Sigma_0 \exp\left(-\frac{R}{R_\mathrm{d}}\right) \mathrm{sech}^2\left(\frac{z}{z_\mathrm{d}}\right).
\end{equation}
The dark matter halo and bulge are generated using \textsc{mkhalo} as part of \textsc{mkgalaxy} and are both designed to have a spherical density distribution:
\begin{equation}
    \rho(r) = \frac{\rho_0}{x^\gamma_i(x^\eta + 1)^{(\gamma_o - \gamma_i)/\eta}} \mathrm{sech}\left(\frac{r}{r_t}\right),
\end{equation}
with the parameters chosen for the halo to provide a Dehnen-McLaughlin profile \citep{dehnen:05} which has the advantage of being fully analytic with a smooth transition between inner and outer parts of the distribution. It also matches very well to simulated dark matter halos. The bulge is a much smaller component of our simulations and therefore we use a standard Hernquist profile \citep{hernquist:90}.

Starting from these initial conditions we change only the mass of the dark matter halo. We create eleven different setups with halo masses ranging from $1.7\times 10^11\ \mathrm{M}_\odot$ to $1.02\times 10^12\ \mathrm{M}_\odot$ which corresponds to the ratio of the radial force from the halo to the disc, $F_\mathrm{h}/F_\mathrm{d}$, ranging from around 0.5 to 3.2 at $R = 8$ kpc and $z = 0$. In other words, we go from disc-dominated systems to halo-dominated systems. Each dark matter halo setup is also regenerated with ten different random seeds to estimate stochasticity, resulting in a total of 110 simulations.

\section{Evolution of non-axisymmetric features}\label{sec:p1-evolution}

\section{Quantifying radial migration}\label{sec:p1-quantifying}