\chapter{The Milky Way as a Galaxy}\label{chap:milkyway}
\begin{flushright}
\textit{``I know, it doesn't make any sense, I wish there were sense.}
\end{flushright}
\begin{flushright}
Where did all the sense go?"

- Tribore, Final Space
\end{flushright}
\section{Components}\label{sec:components}
\begin{figure}[t]
    \centering
    \includegraphics[width=1\textwidth]{images/gaiasky.png}
    \caption{All-sky view of the Milky Way Galaxy from Gaia based on measurements of nearly 1.7 billion stars. We mark the location of different components of the Galaxy with different colors. Image adapted from \textit{Gaia} Data processing and Analysis Consortium (DPAC) (\href{https://creativecommons.org/licenses/by-sa/3.0/}{CC BY-SA 3.0 IGO}).} % Fig. 1.1
    \label{fig:gaiasky}
\end{figure}
All throughout history mankind has sought to understand the night sky and its many features such as stars, clumps, planets or `wanderers'. But no feature is as large and noticeable as the great spray of stars that make up the Milky Way Galaxy, named so for the milk from Hera's breast in Greek mythology \citep{leeming:98}. The suggestion that this milky band of stars was a rotating body that we, the observers, are inside of did not come until \cite{wright:1750}. Since then our understanding of our home Galaxy has increased tremendously and we can present stunningly detailed views of it like the map shown in Fig. \ref{fig:gaiasky}. This map is made possible thanks to measurements from \textit{Gaia}'s second data release (\citealt{dr2},  hereafter DR2). As the figure shows, the Milky Way is composed of several different components with stars differing in spatial distribution, kinematics, chemistry, and age. In the three papers in this theses I touch upon almost every component mentioned in Fig. \ref{fig:gaiasky} and so I will briefly provide a description of each one.

\subsection{Thin disc}\label{subsec:components-thindisc}
The thin disc is what visually makes up the Milky Way and, since it is where the Sun is located, it is the the most well-studied of the stellar components. The thin disc is also the site of ongoing star formation which recent estimates place as high as ${\sim}3.3\ \mathrm{M_\odot yr}^{-1}$ \citep{zari:22}. As the name suggests it is relatively thin with a scale length of $R_\mathrm{t} \approx 2.6$ kpc, scale height of $z_\mathrm{t} \approx 300$ pc, and with a mass $M_\mathrm{t} \approx 3.5\times 10^{10}$ M$_\odot$ \citep{bland-hawthorn:16}. The thin disc stars are generally young and have abundances of elements produced by the $\alpha$-process \citep{burbidge:57} relative to iron similar to the Sun. These elements have nuclei which are multiples of four, the atomic mass number of the Helium nucleus or $\alpha$-particle. We measure the relative abundances relative to iron as:
\begin{equation}
    [\alpha/\mathrm{Fe}] = \log_{10}\left(\frac{N_\alpha}{N_\mathrm{Fe}}\right)_\mathrm{star} - \log_{10}\left(\frac{N_\alpha}{N_\mathrm{Fe}}\right)_\odot,
\end{equation}
where $N$ is the number of atoms per unit of volume of the respective species.

\subsection{Thick disc}\label{subsec:components-thickdisc}
The second disc of the Galaxy fulfills its name with a scale height of $z_\mathrm{T}\approx 900$ pc, scale length $R_\mathrm{T} \approx 2$ kpc, and mass $M_\mathrm{T} \approx 6$ M$_\odot$ \citep{bland-hawthorn:16}. Its stars are older \citep{bensby:14,martig:16} and kinematically hotter than those of the thin disc, since age and velocity dispersion are correlated (\citealt{martig:14,aumer:16}). In metallicity space, thick disc stars occupy regions of higher [$\alpha$/Fe] and are usually linked to the high-$\alpha$ sequence \citep{katz:21}. How the thick and thin discs formed is still a debated topic, particularly so the former as explained in \cite{helmi:20} who also shows that the formation may be related to the evolution of the stellar halo through mergers with nearby galaxies. 

\subsection{Stellar halo}\label{subsec:components-stellarhalo}
The most extended stellar component is the stellar halo which contains $1.3^{+0.3}_{-0.2} \times 10^9$ M$_\odot$ within $2 < r < 70$ kpc \citep{mackereth:20} and is host to the oldest and most metal-poor stars in the Galaxy (e.g., \citealt{dacosta:19, horta:22}). The orbits of halo stars are less flattened towards the disc and so can be told apart locally by their kinematics. Relative to the discs, the halo stars appear to move with a typical velocity of ${\sim}$200 km s$^{-1}$. A commonly used cosmological model is $\Lambda$ cold dark matter ($\Lambda$CDM) which successfully explains, for example, the cosmic microwave background \citep{planck:20} and large-scale structure of galaxies \citep{springel:05}. This theory also explains how stellar halos can form through hierarchical growth with minor and major mergers \citep{white:78, fall:80}. This view matches well with current understanding of the stellar halo as having an \textit{in situ} component of stars as well as an accreted component which becomes extremely dominant at larger distances from the disc \citep{naidu:20}. It has also been shown that this accreted component has a plethora of substructures in it attributed to various accreted stellar populations (e.g., \citealt{koppelman:19, feuillet:21, dodd:22}). We will touch more upon this in sections \ref{sec:p3-gaiaview} and \ref{sec:p3-structures}.

\subsection{Dark matter halo}\label{subsec:components-darkhalo}
There is another halo which is not visible to our telescopes. If we only look at the stellar matter of a galaxy like the Milky Way, the rotational velocity of stars in its outer parts is expected to decrease with distance in a similar fashion to Keplerian rotation, in which $v_\mathrm{rot}^2 \propto M/R$. This is not what we observe however, and instead the rotation curve flattens out which is attributed to the existence of a dark matter halo. Current results place the mass of the dark matter halo at $M_\mathrm{dh} \approx 1.3 \times 10^{12}$ M$_\odot$ \citep{posti:19} and its shape is still a topic of much debate as explained in \cite{mcmillan:17}. While the debate is ingoing, it is very common in simulations to assume a spherically symmetric halo (e.g., \citealt{andersson:20}).

\subsection{The bulge}\label{subsec:components-bulge}
In the central regions of the Galaxy lies the bulge, heavily obscured by dust as is clearly visible in Fig. \ref{fig:gaiasky}. The original explanation of the bulge was that of a spherical structure built up through early mergers, which is called a \textit{classical bulge}. This also made sense given the old ages of bulge stars known at the time \citep{clarkson:08}. More recent results has shown that the bulge has several different metallicity populations \citep{ness:13a} including young metal-rich stars \citep{ness:14}. These younger stars instead suggests that the bulge could have formed through internal disk instabilities. Other works have also shown that parts of the bulge can be formed from the disc through interactions that slowly rearrange energy, angular momentum, and mass, otherwise known as \textit{secular evolution} \citep{kormendy:13}. Following star counts in the bulge, it has been established that the majority of bulge stars participates in a \textit{box/peanut}-shaped structure, related to the three-dimensional Galactic bar, with cylindrical rotation (\citealt{wegg:13, ness:13b}) and \cite{shen:10} uses the kinematics to constrain its contribution to be less than 8\% of the disc mass. In the light of these discoveries, it is unclear if the Milky Way even has a classical bulge.

\section{The bar \& spiral arms}\label{sec:barspirals}
\begin{figure}[t]
    \centering
    \includegraphics[width=.65\textwidth]{images/simgal.png}
    \caption{An example of a simulated Milky-Way like disc Galaxy with pronounced spiral arms and a central bar.} % Fig. 1.2
    \label{fig:simgal}
\end{figure}
Beyond the components mentioned in the previous sections, there are the non-axisymmetric features. In other galaxies they are clearly visible but since we reside inside our Galaxy we struggle to see them as clearly. As an example we show a simulated galaxy in Fig. \ref{fig:simgal} which has a bar and two spiral arms that can be seen very clearly and represents a typical disc galaxy. Since non-axisymmetric features play an important role in secular evolution we will take a closer look at these features in the Milky Way.

The boxy/peanut shaped bulge mention in section \ref{subsec:components-bulge} is an inner, vertical extension of the Galactic bar \citep{bland-hawthorn:16}. The bulge region reaches to about ${\sim}2$ kpc \citep{wegg:13} while the bar may reach as far as 5 kpc \citep{wegg:15}. For this reason it is sometimes referred to as the `long' bar. Current estimates for the bar puts its mass at ${\sim}1.6\pm 0.3 \times 10^{10}$ M$_\odot$ \citep{kipper:20} and using Gaia's third data release (\citealt{dr3}, hereafter DR3) the bar angle with respect to the Sun-Galactic Centre (GC) is estimated to be $-19.2^\circ \pm 1.5^\circ$ \citep{dr3:asymmetries}. The bar is not static however and is rotating with a specific angular velocity, called its pattern speed. The pattern speed of the bar is subject to much debate with many attempts at determining it. \cite{bland-hawthorn:16} review many of the estimates and conclude with an estimated pattern speed of $\Omega_\mathrm{b} \simeq 43 \pm 9$ km s$^{-1}$ kpc$^{-1}$. More recent estimates place the pattern speed of the bar at $\Omega_\mathrm{b} = 33.29 \pm 1.81$ km s$^{-1}$ kpc$^{-1}$ \citep{clarke:22} and $\Omega_\mathrm{b} = 41 \pm 3$ km s$^{-1}$ kpc$^{-1}$ \citep{bovy:19, sanders:19}, in agreement with the previous value. Pattern speeds of this scale has been called a `slow' bar scenario.

The other major non-axisymmetric feature of the Milky Way are the spiral arms. They likely wind around the whole disc and as such, we do not have a full picture of them to date and instead must look to whatever parts of them are visible to us from our position as observers in the Galactic plane. Current evidence within the community is that the Milky Way has four approximately symmetric spiral arms \citep{reid:19} rather than just two. The names for these four arms as in literature are \textit{Perseus}, \textit{Sagittarius-Carina}, \textit{Scutum-Centaurus}, and \textit{Norma-Outer}. The Sun is believed to lie between \textit{Perseus} and \textit{Sagittarius-Carina} in an inter-arm region. In addition to these arms, very close to the Sun lies the \textit{local arm}, initially believed to be a spur of the \textit{Perseus arm}. It has since been understood to be more similar to the nearby arms with comparable qualities, and is perhaps a branch of one of them \citep{xu:13}. In spiral galaxies the highest densities of gas and stars lie along the arms which is the site for most star-formation in the disc. 

\section{Radial migration}\label{sec:migration}
\begin{figure}[t]
    \centering
    \includegraphics[width=.9\textwidth]{images/alphaoverfe.pdf}
    \caption{An illustration of the distribution of abundance of $\alpha$-elements vs the abundance of iron. The dotted line shows the location of the Sun and the gray solid line the expected evolution of an isolated region of the ISM.} % Fig. 1.3
    \label{fig:alphaoverfe}
\end{figure}
Since the spiral arms and the Galactic bar are such very prominent features of our Galaxy and in other similar spiral galaxies, it is no surprise that they have a profound impact upon the disc in which they are found and the stars that live therein. One such effect which occurs because of the dynamical interplay between disc and the non-axisymmetric features is \textit{radial migration}, which is the displacement of a star in the radial direction from the Galactic plane. We will soon explain the major processes which cause radial migration but first the importance of radial migration as an ingredient of galaxy evolution, and the evidence to support it, should be discussed.

Let us consider the chemical evolution of the Galactic disc. The abundances of $\alpha$-elements and Fe over time in an isolated region of the interstellar matter (ISM) are affected by the life and death of its stars through what is called stellar nucleosynthesis (for a review see \citealt{edvardsson:1993}). In short, stars create elements and enhance the abundances of the next generation. Initially core-collapse supernovae produce similar amount of $\alpha$ and Fe, but [Fe/H] increases. Eventually type Ia supernovae begin, which produces Fe but no $\alpha$-elements and thus for the region [$\alpha$/Fe] starts to drop. This behaviour produces a trend like the gray line seen in Fig. \ref{fig:alphaoverfe}. If we are in a very isolated region, we would expect to see that the stars follow this narrow trend. Neighbouring regions radially inside and outside of the region would however have higher and lower [Fe/H] ranges as it has been shown that [Fe/H] increases radially inwards in the Galaxy \citep[e.g.,][]{hayden:15}. In observations of the Solar neighbourhood (e.g., \citealt{edvardsson:1993, hayden:15, bensby:14}) we see a range of different Fe abundances at each [$\alpha$/Fe], similar to the illustration shown in Fig. \ref{fig:alphaoverfe}. Similarly, the age-metallicity relationship (AMR) can be expected to follow a narrow line for an isolated region, but shows a wide scatter. This can be quite easily explained if the different regions of the disc are not isolated from each other, but rather there is radial mixing between them. 

Beyond this rather straight-forward example of radial migration it has been suggested that several other observed features of the Milky Way disc are caused by it. These include the observed bimodality of $[\alpha/Fe]$ in plots like Fig. \ref{fig:alphaoverfe} \citep{schonrich:09, toyouchi:2016} and the flaring of the outer disc in mono-age populations \citep{minchev:12}. It is clear that some form of radial migration occurs in the Milky Way and we are today able to estimate the radial displacement of individual stars, such as in \cite{frankel:18} who find that the Sun has likely migrated from a birth radius of {$\sim$}5.2 kpc. Therefore it is important to understand the processes by which migration occurs.

\subsection{Radial heating}
\begin{figure}[t]
    \centering
    \includegraphics[width=1\textwidth]{images/radialmigration.pdf}
    \caption{A simple sketch of the radial evolution of an orbit that undergoes migration. \textit{Left}: The effect of radial heating or blurring, i.e. the increase of random motion due to dynamical scattering. This process increases or decreases the amplitude of the radial oscillations of the orbit. The shaded regions mark the time during which the radial heating occurs. The average radius, or guiding center radius, $R_g$, is never changed during the process. \textit{Right}: The effect of cold torquing or churning by scattering at a corotation resonance, which displaces the guiding center radius, $R_g$, but does not increases the amplitude of oscillations and therefore does not increase the radial action, $J_R$.} % Fig. 1.4
    \label{fig:radialmigration}
\end{figure}

One rather simple cause of radial migration is what is called \textit{radial heating}, sometimes called \textit{blurring}. Stars are born in Giant Molecular Clouds (GMCs) which move on nearly circular orbits around the disc. This means that the stars themselves are born on nearly circular orbits. But through the evolution of stellar orbits they can scatter by interaction with things like other GMCs, clusters, or spiral arms which will lead them onto slightly eccentric orbits, called \textit{epicycle orbits} as they can be described by the \textit{epicycle approximation}. The epicycle refers to the radial and azimuthal oscillations of the perturbed orbit, occurring with an \textit{epicycle frequency}, $\kappa$. The reason a perturbed star does not simply move to a different radius when scattered is because of the fine balance between centrifugal and gravitational force keeping it in place. If the star is pushed radially outwards, the centrifugal force decreases faster than gravity and the star moves back in. The star now overshoots to an interior radius where the centrifugal force increase faster than gravity which pushes it back out. In other words we say that the star is stable to small velocity changes. Because of the oscillations the star will visit different radii than its original radius, called the \textit{guiding radius}, $R_g = L_z / v_c$, where $L_z$ is the angular momentum perpendicular to the disc and $v_c$ is the circular velocity. It is the process of increasing the amplitude of the oscillations that we call radial heating and we show how this might look in the left panel of \ref{fig:radialmigration}. The scattering process will not change the guiding radius on average but in the cases in which it does, this increases the random motions of the star, including the amplitude of the radial oscillations. In short, while the process of radial heating does not directly relate to a change in guiding radius, such a change can in certain instances also occur. 

Given that the stars visit other regions of the Galaxy, they can obviously enrich those regions as well which leads to the conclusion that radial heating can contribute to the width observed in the chemical evolutionary tracks discussed in the previous section. It can however be shown, as in \cite{binney:07}, that radial heating will only account for around 50\% of the observed scatter in the metallicity and instead there must be some additional source of mixing to explain the measured scatter. 

\subsection{Cold torquing}
\begin{figure}[t]
    \centering
    \includegraphics[width=0.8\textwidth]{images/torquing.pdf}
    \caption{The concept of horseshoe orbits and angular momentum transfer near corotation of a non-axisymmetric feature with constant angular speed. The gray bar is a non-axisymmetric feature, with the Galactic centre towards its bottom. The dashed line marks the corotation radius and the pink points are positions along the horseshoe orbit right before angular momentum transfer. The vectors for $\bm{r}$ and $\bm{F}$ which gives the torque is indicated} % Fig. 1.5
    \label{fig:torquing}
\end{figure}
Another source of radial mixing was described first in a seminal paper by \cite{sellwood:02} where it was shown that disc heating is not the dominant effect of the spiral arms. Instead, non-axisymmetric features like the bar and spiral arms are able to shift the guiding radii of stars without significantly altering their dynamics. This process occurs through resonant interactions with the non-axisymmetric features. The spiral arms or bar will exert a torque that changes the angular momentum of the star's orbit since:
\begin{equation}
    \frac{d}{dt}\bm{L} = \frac{d}{dt}(\bm{r}\times\bm{p}) = \bm{r}\times\bm{F} = \bm{\Gamma}.
\end{equation}
Axisymmetric features like bars and spirals move with constant angular velocity, which means that the non-angular velocity increases further out. This means that for stars with approximately constant circular velocity there is a radius at which the velocity of a star and spiral/bar is the same, called \textit{corotation}. Beyond this point stars move more slowly than the spiral and within it they move faster. Faster stars catch up to the feature and will have a force, $\bm{F}$, directed towards it. Slower stars instead fall into it with a force in the opposite direction. We illustrate this in Fig. \ref{fig:torquing} which shows that for the fast stars, the torque will be directed inwards, i.e., negative which decreases the angular momentum and transfers it to a smaller $R_g$ orbit where it moves faster. It eventually catches up to the spiral/bar and is given a positive torque, migrating outwards. In this simple view the orbit would go back and forth but due to the transient nature of spiral arms and the plurality of axisymmetric features, this is not a likely outcome.

\begin{figure}[t]
    \centering
    \includegraphics[width=0.6\textwidth]{images/actionangle.pdf}
    \caption{A polar coordinate analogy of action-angle variables to phase-space coordinate. An orbit oscillating in some coordinate $q_i$ will have a velocity, $v_i$ which also oscillates. A point on the same orbit can be described with a constant area, $2\pi J_i$, since the action $J_i$ is a conserved quantity, and an angle $\theta_i$.} % Fig. 1.6
    \label{fig:actionangle}
\end{figure}

This provides the `torquing' part. The cold part of the term comes from the fact that the migration does not increase the random motion of the affected stars, leaving them kinematically unscathed from the interaction. To properly explain this concept, we introduce \textit{action-angle} variables to describe a stellar orbit. A simple polar-coordinate analogy is given in Fig. \ref{fig:actionangle} which shows that an orbit in a certain dimension, $i$, can be described with two oscillating phase-space coordiantes $(q_i, v_i)$ or one constant and one oscillating, $(J_i, \theta_i)$. For disc orbits the three actions usually used are: $J_r$ which describes the extent of the radial motion, $J_z$, the extent of the vertical motion, and $J_\phi$ which is equivalent to $L_z$. 

If we place ourselves in the rotating frame of a Galaxy, moving with a pattern speed $\Omega_p$ relative to the intertial frame, we can describe its total energy or Hamiltonian, $E_J$, using the Hamiltonian of the inertial frame, $E$:
\begin{equation}\label{eq:jacobi}
    E_J = E - \Omega_pL,
\end{equation}
Where $L$ is the angular momentum. We call $E_J$ the \textit{Jacobi integral} (see \citealt{binney:08} chapter 3.3.2 for a full derivation) and it is constant in time. So the difference in energy between two points in time is:
\begin{equation}\label{eq:deltaE}
    \Delta E = \Omega_p\Delta L.
\end{equation}
Now we consider the energy separated into a radial and an azimuthal part
\begin{equation}\label{eq:Echange}
    \Delta E = \frac{\partial H}{\partial L}\Delta L + \frac{\partial H}{\partial J_R}\Delta J_R,
\end{equation}
where $J_R$ is the radial action. If we use Hamilton's equations with $\bm{J}$ as the momentum and $\bm{\theta}$ as the coordinate we find:
\begin{equation}
    \dot{J}_i = - \frac{\partial H(\bm{J})}{\partial \theta_i} = 0 \qquad \dot{\theta}_i = \frac{\partial H(\bm{J})}{\partial J_i} = \Omega_i(\bm{J}).
\end{equation}
where $\Omega_i$ is an angular velocity. This can be understood considering the total energy or Hamiltonian of an orbit is a function of the action, but not of the angle; the energy does not change during an orbit. If the actions, which describe velocities and positions, changes then of course so must the energy. We have that $\partial H / \partial J_r = \omega_R$ and $\partial H / \partial L = \Omega$, the radial and azimuthal frequencies. Combining this with eq. \eqref{eq:Echange} gives
\begin{equation}
    \Delta E = \Omega\Delta L + \omega_R \Delta J_r,
\end{equation}
which when put into eq. \eqref{eq:deltaE} yields
\begin{equation}\label{eq:churning}
    \Delta J_r = \frac{\Omega_p - \Omega}{\omega_R}\Delta L.
\end{equation}
It is the implications of eq. \eqref{eq:churning} that provides the `cold' part. 

Stars have almost constant circular velocity across the disc which means that angular momentum corresponds to guiding radius, (approximately, $L_z \propto R_g$). We know that torquing provides a change in the angular momentum from our discussion above, so this will correspond to a change in the radial action of the orbit, unless the angular velocity of the star matches that of the axisymmetric feature, $\Omega = \Omega_p$, which is the case near corotation. This behaviour was demonstrated and detailed in \cite{sellwood:02} which showed also that the migration caused by the cold torquing can displace the star on kiloparsec scales, without imparting any increased random motions. The nature of cold torquing is therefore not only impressive but also frustrating since if stars are able to change their guiding radius by large scales without having any kinematic evidence of such a process, the history of different regions of the Galaxy become much more complex. 

As an aside, it is worth mentioning the influence of other resonances that exist besides corotation. A particularly strong such resonance is when the radial frequency, $\omega_R$ (identical to $\kappa$ in the epicyclic approximation), is a multiple of the frequency with which the star encounters the non-axisymmetric feature $(\Omega_p - \Omega)$. In other words when
\begin{equation}
\omega_R = \pm m(\Omega_p - \Omega).
\end{equation}
These resonances are called \textit{Lindblad resonances} after Swedish astronomer Bertil Lindblad (1895 - 1965). The value of $m$ is set by the symmetry of the perturber. For example $m=2$ is a two-armed spiral or bar and $m=4$ a four-armed spiral. The positive sign corresponds to an orbit in which the rotating feature sweeps by the slower star as it completes $m$ radial oscillations and is known as an \textit{Outer Lindblad Resonances} (OLR). The negative sign is when the fast star rotates past the rotating feature by the time it completes its $m$ radial osscillations and is then called an \textit{Inner Lindblad Resonance} (ILR). At these resonances we have from eq. \eqref{eq:churning}
\begin{equation}
    \Delta J_r = \pm \frac{1}{m}\Delta L,
\end{equation}
which shows how migration at these resonances increases the radial action, therefore making them a potential source of migration by radial heating as discussed previously.

Clearly, we must understand how stars are radially migrated. In particular, the process of cold torquing must be well understood since it is unique in the fact that it leaves no dynamical trace. Without these insights we cannot have a full picture of the history and evolution of our Galaxy.