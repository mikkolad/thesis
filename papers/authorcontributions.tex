\sectionhidenum{Paper summaries and author contributions}

A summary of each paper is presented here, and my contributions clarified.

\subsection*{Paper I}

In this paper we have gone through the first part of our dataset taken from the first observation trip. In the dataset we found a metal poor star in the hotter end of M giants with an effective temperature of about 3800 K. This means that its spectrum was devoid of most of the molecular features that we find in the other stars in our dataset. Hence, the star's spectrum was easier to analyse.

Careful orbit analysis of the star suggests that the star is a member of either the nuclear star cluster (NSC) or the nuclear stellar disk (NSD). The orbit does not show signs of it being a stray star from the bulge that has been captured, though we cannot rule out that the star may originally have formed in a globular cluster that had fallen into the Galactic centre.

That the star is metal poor, [Fe/H] $\sim -1.0$, meaning that it could be similar to globular cluster stars. Other studies of the Galactic centre are finding 5--10\% low metalliticy stars \citep{Feldmeier-Krause2017,do:15}, so our one star out of the 20 in our sample agrees with that.

We find that [\textalpha/Fe] $\sim 0.4$, which is similar to what is found elsewhere in our Galaxy for stars with the same metallicity. The study of this one star does not give reason to claim any differences between the Galactic centre and other parts of the Galaxy.

\subsubsection*{My contribution to paper I}

I took part in the observations, having led some of the time. I was responsible for the data reduction. I developed the astrophysical line list used for the spectroscopy analysis and carried out the abundance analysis in close collaboration with Nils Ryde. I contributed to writing the article.


\subsection*{Paper II}

In this paper we present a metallicity distribution function (MDF) of stars found in the nucleus of our Galaxy, either NSC or NSD. The stars seem to show two peaks in the MDF, one centred around [Fe/H]~$\sim -0.25$ and one centred around [Fe/H]~$\sim 0.4$.

Because of patchy extinction towards the Galactic centre, it turned out to be really difficult to get proper photometric $\log g$ values for all of the stars in the dataset. Given that we have targeted old stars we we were able to make use of the fact that isochrone tracks for M giants are insensitive to age for ages above 5\,Gyr. This means we have a function linking the three stellar parameters: effective temperature, [Fe/H] and $\log g$. Performing an iterative metallicity determination including the constraints from this developed function made it possible to determine the metallicities of the stars.

This method was verified against a subsample of giant stars in the APOGEE data set having effective temperatures between 3500 K and 4500 K so as to be as comparable to the M giants in our sample as possible. We also did the analysis using the photometric $\log g$ determinations, even though we had problems with the patchy extinction. The photometric results are in good agreement with the isochrone $\log g$ for many of the stars but, as expected, exhibit larger deviations for some of the stars.

We investigated many of the strong spectral lines belonging to scandium, titanium, vanadium and calcium. However, we found many lines to be very temperature sensitive and further suspected NLTE to be an issue, so we did not trust our abundance analysis of these lines.

\subsubsection*{My contribution to paper II}

I took part in the observations, having led some of the time. I was responsible for the data reduction. I used the previously developed astrophysical line list and did the abundance analysis in close collaboration with Nils Ryde. I developed the isochrone method required for finding the metallicity distribution. I contributed to writing the article.


\subsection*{Paper III}

After our paper II had been published, a publication \citep{do:18} appeared claiming unusual scandium, vanadium and yttrium abundances in the Galactic centre. We thus decided that it was necessary to return to the strong spectral lines we had ignored in Paper II and reconsider our data.

Our dataset also includes observations of M giants further out in the Galaxy that we use as a control sample. This allowed us to make a differential studies, where a lot of the systemic uncertainties are eliminated.

Comparing our Galactic centre stars and stars further out in the Galaxy, we found that these strong lines are also present further out in the Galaxy. Therefore, the strong lines are likely to be intrinsic to the cool stars that we are probing and not due to the stellar population or environment where the stars live.

We took a closer look at scandium, since we had recently measured its atomic properties in the lab. Many of the atomic physics properties of scandium were at the time not included in the database VALD3, and including them actually could explain partially the strong spectral lines. In particular temperature sensitivity due to ionisation and the hyperfine splitting properties of scandium are both significant contributors. Further investigation of the temperature sensitive NLTE effects will be needed to enable determination of Sc abundance from these lines.

\subsubsection*{My contribution to paper III}

I took part in the observations, having led some of the time. I was responsible for the data reduction. I used the previously developed astrophysical line list and isochrone method and carried out the Sc abundance determinations. I modelled scandium with and without hyperfine structure. I led the paper writing.


\subsection*{Paper IV}

This paper is unusual in that it started as a conference proceeding but, for some reason, it was assigned no less than three referees. Thanks to the comments from the referees it took off from there to become a work that focused on making the astrophysical results from the previous articles available to the atomic physics community.

\subsubsection*{My contribution to paper IV}

I was wholely responsible for this paper.


\subsection*{Paper V}

Since the publication of Paper II we have concentrated on finding good alpha lines to investigate. There are several good silicon lines and, after doing a thorough investigation into identifying molecular features of cool stars, we were able to compile a good set of silicon lines that could be used for abundance analysis.

After we had demonstrated the temperature sensitivity of scandium lines in Paper III we used this to implement a temperature determination function, rather similar to the CO bandhead one we have used in Paper II. It turned out that the temperatures determined from the scandium lines were generally in good agreement with the temperature from the CO bandhead method, which has increased our confidence in the determined effective temperatures of the stars.

Using the updates on molecular blending and the slightly improved temperatures, we redid the MDF analysis. It is quite similar to the MDF we found in Paper II, although the new MDF shows even more strongly the bimodality in the MDF suggested in Paper II.

We also investigated the silicon abundance as a tracer element for alpha elements and we find, curiously enough, a slightly enhanced silicon abundance for the metal-rich end of our dataset. We investigated possible systematics to be confident with the results. NLTE calculations showed that NLTE corrections are not a significant concern for our choice of silicon lines.

We entered into a collaboration with the community working on chemical evolution models and, in this paper, offer two possible explanations for the enhanced silicon abundances in metal-rich stars.


\subsubsection*{My contribution to paper V}

I took part in the observations, having led some of the time. I was responsible for the data reduction. I developed a scandium temperature determination method to improve on our temperatures. I used the previously developed astrophysical line list and isochrone method and performed the updated metallicity distribution and the abundance analysis. I investigated line blending by modelling theoretical spectra to minimise the impact of CN blending. I led the paper writing, including the coordination of contributions from the new collaborators.

