\newpage
\subsection*{Paper I: Radial migration and vertical action in N-body simulations}
\textbf{Mikkola, D}.; McMillan, P. J.; Hobbs, D. (2020) \newline
Monthly Notices of the Royal Astronomical Society, Volume 495, Issue 3, pp. 3295-3306 \newline

\subsubsection*{My contribution:}
Paul McMillan (PM) had the original idea for the paper after reading about the role of migration for large vertical excursions in \cite{solway:12} and \cite{vera-ciro:14, vera-ciro:16b}. Daniel Mikkola (DM) set up and ran the numerical simulations and wrote all of the analysis tools that were used. David Hobbs (DH) and PM advised on choice of simulation code and on what analysis to perform. Outputs of simulations were analyzed by DM with guidance by his supervisors DH and PM and discussed by DM, PM, and DH. PM provided literature resources for the writing and DM wrote and submitted the manuscript. Several rounds of revisions, feedback, and discussions between the authors as well as with an anonymous referee led to the final product which was accepted for publication.\newpage


\subsection*{Paper II: The velocity distribution of white dwarfs in Gaia EDR3}
\textbf{Mikkola, D.}; McMillan, P. J.; Hobbs, D.; Wimarsson, J (2022) \newline
Monthly Notices of the Royal Astronomical Society, Volume 512, Issue 4, pp. 6201-6216 \newline

\subsubsection*{My contribution:}
The original idea came from PM to apply the method of \cite{dehnen:98a} to DR2 \citep{dr2}. John Wimarsson (JW) wrote the initial, core parts of the code with support from DM and PM. DM took over responsibility for the code, finished writing it as well as developed all of the analysis tools. Further developments for the code were implemented with input from PM and DH, including the multigrid approach. DM and PM chose white dwarfs as targets and DM acquired the data from the Gaia archive. DM carried out the analysis with guidance from his supervisors DH and PM. DM reviewed existing literature on velocity distributions as well as white dwarfs and their bifurcation. DM wrote the paper, with rounds of review between DM, PM, and DH. DM submitted the report and went through editing rounds with an anonymous referee's useful comments which lead to further clarity and discussion in the paper. The paper was then accepted for publication.\newpage


\subsection*{Paper III: New stellar halo substructures from Gaia DR3 proper motions}
\textbf{Mikkola, D.}; McMillan, P. J.; Hobbs, D. (to be submitted to MNRAS) \newline

\subsubsection*{My contribution:}
After the second paper, DM and PM had the original idea to expand the method and apply it to other data sets. DM and PM chose two new data sets which were a large Solar neighbourhood sample and a stellar halo sample. PM provided literature sources for velocity distributions of the stellar halo. DM acquired the data and filtered it for quality. DM implemented conversion into spherical velocity coordinates in the code and ran the algorithm to produce probability distributions. The analysis in velocities was carried out by DM, with support from PM and DH and minor contributions from PM. DM wrote the paper which was reviewed by PM and DH to bring it to a submittable state.
\newpage




