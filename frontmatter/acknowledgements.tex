\sectionhidenum{Acknowledgements}
I am forever grateful to who is probably my greatest supporter, my beloved partner Tina Sörensen who has been by my side ever since I started my doctoral journey. Your unwavering support has been with me through difficult times with debugging, writing, contemplating, and a global pandemic. I would not be where I am today without you. 

My supervisor, David Hobbs. You have mastered the art of knowing when to give a push and when to encourage letting go and taking a step back. You have inspired my work ethic and taken the best care of me as a student of yours. Your door has always been open to me and so has your ear. I will miss walking around the corner to have a chat about just anything.

My co-supervisor, Paul McMillan. Your guidance has been invaluable to me ever since we started working together during my master's project. I am ever impressed by your intuition when looking at new results and to me you have always been a bottomless fount of knowledge. I regret that this is our last project together, however I think I now know what a good enough scientist would do. 

I want to thank the friends I have made at Lund Observatory during my stay here. There are too many master students, fellow PhDs, and staff members I want to thank for me to name them all here. Especially I want to thank Eric Andersson who's thesis has been in tandem with mine, which lead to good friendship over the years. My first and second office-mates Iryna Kushniruk and Bibiana Prinoth, my closest colleagues in almost all matters besides research. You were the ones I could always `turn' to and have a chat about either work of life in general. Wherever I end up in the future I am certain I will not have such a great office situation as the one you have given me. 

I must not forget to thank the Physics \& Lasershow. For almost ten years the show has let me mix work with incredible amounts of fun and given me some of my closest friends. Thank you Per-Olof, Johan Z, Stina, Odd, Johan K, Jonas, Alexandra, Frida, Rebecca, Vassily, Matteus, Lina, David, Elin, and Anna-Maria.  

My friends outside of work: Anton, Sara, Fredrik, Marielle, Rasmus, Anna, Jesper, Lovisa, Johan. You have always made sure to keep me humble and to ask me any and all \textit{astrology}-related questions, much to my bemusement. Adrian, I am thankful we reconnected during our PhDs and could share many hours working out or betraying each other in board games.

To conclude, I want to extent my deepest gratitude to the people who made sure I got here today: my family. First, my parents Marita and Seppo, who always made sure to let me explore while providing all the support I needed. My bonus-mother Suzie, I aspire to have your fortitude. My five brothers Nicholas, Robert, Christopher, Mattias, and Alexander. Of course also my extra family in Halmstad who have always welcomed me. 

%I’m deeply indebted to my lovely wife, Kristina Arnrup Thorsbro, without whom this thesis wouldn't exist. Quitting my well paid job and becoming a student is not a decision you take lightly. Kristina did not so much as flinch at my crazy idea and has been at my side and supporting me every step of the way on this journey.
%
%I would also like to extend my deepest gratitude to my supervisor Nils Ryde, who with his extreme generosity has included me in his scientific collaborations from day one and has trusted me to lead the work on several occasions. Nils Ryde and my co-supervisors, Hampus Nilsson and Henrik Jönsson, have deep know\-ledge of the field, which they have shared willingly and patiently in spite of my constant questioning. For this I am very grateful.
%
%For the writing of my thesis I am deeply indebted to Colin Carlile, Florent Renaud, Rebecca Forsberg and Kasper Brandt Nyegaard; Colin for his meticulous and thorough effort to help me improve my English writing skills, Florent for for being present all summer and willing to discuss all the scientific ideas that I wanted to put in my thesis, Rebecca for helping with such a good translation of my English popular summary into Swedish and Kasper for doing the layout of the front page.
%
%Mike Rich is an inspiration to me with his great enthusiasm for astronomy and extremely wide knowledge of the field. I am especially thankful for our time together at the Keck telescope and being taught how to enjoy life on Hawai'i. I am also very grateful for the time Mike hosted me at UCLA and in his home in Bel Air, Los Angeles---I am looking forward to the next trip.
%
%Mathias Schultheis is also an inspiration to me showing how it is possible to be a great astronomer and still keep both feet grounded on the earth. I am very grateful for the many times that Mathias has hosted me both at the observatory in Nice, but also in his home in Mougins. Who would have thought that it was so productive to sit and work at a beach café at the Côte d'Azur. I am especially thankful for the opportunity to visit and work with Mathias for a month at the observatory in Nice---I can imagine spending more time there!
%
%I am grateful for the collaboration that I have had on the Galactic centre work; in particular Livia Origlia and Tobias Fritz have been along for the entire project and have taught me a lot. A special thanks goes out to Livia for visiting us in Lund and teaching me the ins and outs of the REDSPEC data reduction kit.
%
%The LUMCAS collaboration on laboratory astrophysics with Malmö University has a special place in my time as PhD student. I have greatly enjoyed going almost weekly to Malmö for a fresh change of air, and I am grateful that Per Jönsson, Henrik Hartman, Tomas Brage, Lars Engström and the rest of the crew are so warm and friendly. And yes, let us do that yttrium paper this autumn, it will be great fun!
%
%The Galactic centre project would not have been a reality without the yearly and fantastic work shops in Sexten in the Dolomites, Italy, hosted by Francesca Matteucci and Carlo Morossi. Going there every year, sharing my results and getting incredible feedback has been a corner stone of my work. In particular, discussions at Sexten with Francesca Matteucci and Emanuele Spitoni has provided me with good inspiration.
%
%I am thankful towards Anish Amarsi and Karin Lind for hosting me in Heidelberg, Germany, and teaching me the ways of NLTE calculations. I am also thankful towards Chris Sneden, Greg Mace and Melike Afsar for hosting me at UC Austin, Texas, USA, discussing my work and looking into making future observations with IGRINS a possibility---I hope to get the chance to observe with this instrument.
%
%A special thanks to my friends and colleagues at the Depart of Astronomy and Theoretical Physics. In particular Noemi Schaffer and Eric Andersson for being brilliant office-mates, Lennart Lindegren for being my excellent undergraduate supervisor and Dainis Dravins for sharing his immense depth of knowledge over the years. And a shout out to all the awesome people that have joined for movie nights or a beer at one of the local bars---good times!
%
%Science has not been my only pursuit during my time as a PhD student as I can not help but engage in the organisation I work in, perhaps it is an occupational hazard from being an entrepreneur for more than 15 years. A special thanks goes out to Daniel Michalik for dragging me into student union work, to Andrea Adden for being such a great chair of NDR and teaching me the ropes, to Andrew Lifson, Leif Gellersen and Lea Miko Versbach for taking over after me in NDR. Also a special thanks to Tanya Kolyaka and Liang Zhao for being with me in the LDK presidium, Luis Serratos and Richard Croneberg for their support and Smita Chakraborty and Harsh Shah for having the courage to take over after me. I am very grateful towards all the good and talented people that I have had the opportunity to work with in the student unions. There are indeed many, which gives me hope for the future!
%
%Finally, a special and warm thanks to my family and friends that have always been there for me. It means a lot to me. Jakob Thorsbro, I enjoy your yearly visits to Lund, I always look forward to them.
