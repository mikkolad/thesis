
As the Galaxy evolves over time it is subjected to a plethora of various dynamical processes that leave their mark upon it for some duration of Galactic time. We can view a snapshot of the ongoing processes and connect it to past events. To do this accurate picture of the stellar motions and positions which can be predicted by numerical models is required. This is the two-pronged approach of Galactic dynamics which connects both theory and observations.

This thesis summarizes three papers in which I make use of these two processes to gain new insights into both the current structure of the local Galaxy and one of the processes responsible.

In Paper I, I seek to determine the relationship between the vertical extent of a stellar orbit and its potential to participare in migration radially across the disc. By numerically simulating multiple different types of discs I am able to determine that radial migration primarily affects stars with small vertical excursions in gravitationally weak discs and affects all stars more equally in stronger discs.

In Paper II, I utilize the proper-motion limited catalogue of Gaia EDR3 to determine the most accurate velocity distribution of white dwarfs to date and provide kinematics for the bifurcation visible in the white dwarf cooling sequence. The paper is able to find known and some novel structure in the velocity distribution and finds that the bifurcation is related to two separate kinematic populations of white dwarfs.

In Paper III, I further apply the methods of Paper II to two new samples of stars: the local stellar halo and the Solar neighbourhood. With the updated Gaia DR3 I am able to show accurate velocity distributions for both samples and can reveal new velocity substructures in the local halo which are part of the `accreted ́ part of the stellar halo.