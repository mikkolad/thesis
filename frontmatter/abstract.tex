Clues to galaxy evolution from spectroscopic observations of Galactic centre stars

In this work we present results from spectroscopic observations of Galactic centre stars. High resolution stellar spectroscopy can be used to determine accurate stellar metallicities and abundances. Observing stars in the Galactic centre is challenging due to extreme extinction. However, observing bright M giants in the K band is viable with 10 m telescopes, which is what has been carried out in this work using the Keck II telescope at W. M. Keck Observatory, Hawai'i.

We provide a metallicity distribution of a sample of stars observed in the Galactic centre and show that the sampled stars on average have a metallicity comparable to the Sun, with a subset of the sample having a very high metallicity. We also investigate the silicon abundance of the stars as an alpha tracer, and show that in general there is a similarity between the Galactic centre stars and stars further out. However, for the high metallicity subsample stars in the Galactic centre, there is evidence for a possible alpha enrichment beyond what is found elsewhere in the Galaxy.

Alpha enrichment is a powerful diagnostic as it is central to chemical evolution models giving constraints important for the development of galactic formation and evolution theories. We model the determined alpha enrichment and suggest that there might have been a recent starburst event, or maybe there was a pause in star formation between 3 and 12 Gyr ago. We model different pause scenarios. Further observations of a larger number of stars, and other tracers of alpha elements, are required to verify this result.

We also investigate claims of increased scandium abundances in the Galactic centre and find that the extremely strong scandium lines could be explained by a better understanding of the atomic physics properties of scandium, rather than a high scandium abundance. We find similarly strong scandium lines in stars further out in the Galaxy.

We have thus demonstrated that the determination of abundances of Galactic centre stars is now possible and that future investigation of more stars and more elements will provide necessary and strong constraints to theories of how the Galactic centre have formed and evolved.
