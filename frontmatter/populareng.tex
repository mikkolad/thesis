\sectionhidenum{Popular summary}
Of all things on the celestial vault the greatest and most striking is the diffuse band of stars that make up the Milky Way, our home Galaxy. To understand how it formed and evolved we need, among other things, a detailed description of how the stars move and where they are located, the field of research called \textit{astrometry}. The first stellar catalogue was created 200 BCE by Hipparchus in ancient Greece. A little over two millennia later we began measuring stars with space telescopes, the first of which was named \textit{Hipparcos}. This was succeeded by the space telescope \textit{Gaia} which was launched in 2013 and revolutionized the field of astrometry, providing a truly great catalogue of stellar motions and positions. This catalogue has significantly contributed to the research behind this thesis. \textit{Gaia} gives us a very precise picture of how the Milky Way kinematics look today. To complete the picture one also uses numerical simulations to recreate and interpret the features found in observations. The union of theory and observations then reinforce one another and is critical for our understanding of the Milky Way.

The first article in this thesis uses numerical simulations to study how interactions with the Galaxy's spiral arms and bar transports stars radially in the plane of the disc. We find that this migration depends on the Galactic disc's strength and how vertical extended the stellar orbit is. With over 100 simulated discs we could determine that in less massive discs it is mostly the stars close to the disc that migrate and in the opposite case of massive discs they migrate regardless of how far above the disc the orbit goes.

In the second and third articles we use data from \textit{Gaia}. By using positions and velocities along the celestial sphere, without needing velocities in the line-of-sight direction, we have access to extremely large amounts of data. This way, we obtain an estimate for the local stellar velocity distribution, despite lacking one velocity component. We do this for three samples of data. In the second article we used white dwarfs, which are the remains of low mass dead stars, and could discover that there were two separate kinematic populations. In the third article we used the Solar neighbourhood of stars in the disc and a local part of the Galaxy's halo. We were able to identify many known structures in the velocity distribution, as well as some new ones which then belong to the accreted halo, and could be the remains of accreted dwarf galaxies. 


% Utav alla ting på himlavalvet så är det största och mest slående det diffusa band av stjärnor som upppgör Vintergatan, vår hemgalax. För att förstå hur den skapades och utvecklas behöver vi, bland annat, en detaljerad bild av hur stjärnorna rör sig och var det befinner sig. Den första stjärnkatalogen kom 200 f.kr från Hipparchos i antika Grekland. Två millenier senare så började vi mäta stjärnor med rymdteleskop, det första då döpt till just Hipparcos. Detta efterföljdes av rymdteleskopet Gaia, som bidrar stort till forskningen bakom denna avhandling. Detta ger oss en väldigt noggran bild av hur Vintergatans kinematik ser ut idag. För att fullända bilden, använder man sig dessutom utav numeriska simulering som kan återskapa de resultat som vi ser i mätdatan. Föreningen av teori och observationer förstärker då varandra. 

% Den första artikeln i denna avhandling använder just numeriska simuleringar för att titta på hur stjärnor förflyttar sig in och ut från Galaxen på grund av interaktioner med dess spiralarmar och centrala stav. Mer specifikt hur denna migrering beror på Galaxskivans styrka och hur vertikalt utsträckt sjärnans omloppsbana går. Med över 100 simulerade skivor kunde vi bestämma att i gravitationellt svaga skivor migrerar mest stjärnorna nära skivan och i starka skivor migrerar de oavsett vertikal omloppsbana. 

% I den andra och tredje artikeln använde vi mätdatan från Gaia. Genom att vi använder positioner och hastigheter tangentiellt på himlen, utan att kräva hastigheter i siktriktningen, hade vi tillgång till extremt stora mängder data. Vi kan då lokalt få en uppskattning av stjärnornas hastighetsfördelning, även fast vi saknar en hastighet. Detta gjorde vi för tre olika urval av data. I artikel två använde vi vita dvärgar, vilket är kvarlevorna av döda stjärnor, och kunde upptäcka att där fanns två separata kinematiska populationer. I artikel tre använde vi Solens kvarter av stjärnor i skivan och en lokal del av Galaxens halo. Där lyckas vi identifiera många kända strukturer i hastighetsfördelningen, samt några nya som då tillhör den ansamlade halon, och skulle kunna vara kvarlevorna av ansamlade dvärggalaxer. 
