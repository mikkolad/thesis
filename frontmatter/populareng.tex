\sectionhidenum{Popular summary}

%Exploring the world we live in is intrinsic to human curiosity. In astronomy this means observing the stars in the Universe around us and trying to understand how it came to be the way it is.

%In the work presented in this thesis we have observed stars in the centre of our Galaxy, The Milky Way. We know from studies of other galaxies, that the centre of a galaxy can tell us something about the rest of the galaxy as a whole. This suggests that galactic centres may have coevolved together with the rest of their galaxy.

%Since the centre of the Milky Way is the closest galactic centre to us it is the one that we can study in the most detail. Assuming the mediocrity principle, i.e.\ that we are not special, the detailed observations of our own Galactic centre and how it is connected to the rest of the Galaxy can provide a better basis for our fundamental understanding of how galaxies are connected to their centres.

%In our work we find both similarities and differences between the stars found in the centre of the Milky Way and stars found further out in the Galaxy. Stars mostly consist of hydrogen and helium, but will be enriched with other chemical species to varying degrees depending on their formation history. Similarly enriched stars exist in both the centre of the Galaxy and further out. For the extremely enriched stars we find a divergence, with the Galactic centre stars containing more silicon than the stars further out in the Galaxy. This suggests that there may have been different formation histories between the two locations.

%If these results can be confirmed by future studies they can be an important part of the bigger question on how our Galaxy formed and evolved. Furthermore, this knowledge can be extrapolated to other galaxies, and perhaps in the end provide important clues to galaxy evolution and formation in general.


