\sectionhidenum{Popular summary}
Of all things on the celestial vault the greatest and most striking is the diffuse band of stars that make up the Milky Way, our home Galaxy. To understand how it formed and evolved we need, among other things, a detailed description of how the stars move and where they are located. The first stellar catalogue came 200 BCE by Hipparchos in ancient Greece. A little over two millennia later we began measuring stars with space telescopes, the first of which then named Hipparcos. This was succeeded by the space telescope Gaia, which contributes a lot to the research behind this thesis. This gives us a very precise picture of how the Milky Way kinematics look today. To complete the picture one also uses numerical simulations that can recreate the results we see in the data. The union of theory and observations then reinforces one another.

The first article in this thesis uses specifically numerical simulations to look at how stars move in and out of the Galaxy because of interactions with its spiral arms and bar. More specifically how this migration depends on the Galactic disc's strength and how vertical extended the stellar orbit is. Wit over 100 simulated discs we could determine that in gravitationally weak discs mostly the stars close to the disc migrate and in strong discs they migrate regardless of vertical orbit.

In the second and third articles we use data from Gaia. By using positions and velocities tangentially on the sky, without needing velocities in the line-of-sight, we had access to extremely large amounts of data. We can then locally get an estimate for the stellar velocity distribution, despite lacking one velocity component. This was done for three samples of data. In article two we used white dwarfs, which are the remains of dead stars, and could discover that there were two separate kinematic populations. In article three we used the Solar neighbourhood of stars in the disc and a local part of the Galaxy's halo. There we were able to identify many known structures in the velocity distribution, as well as some new ones which then below to the accreted halo, and could be the remains of accreted dwarf galaxies. 


% Utav alla ting på himlavalvet så är det största och mest slående det diffusa band av stjärnor som upppgör Vintergatan, vår hemgalax. För att förstå hur den skapades och utvecklas behöver vi, bland annat, en detaljerad bild av hur stjärnorna rör sig och var det befinner sig. Den första stjärnkatalogen kom 200 f.kr från Hipparchos i antika Grekland. Två millenier senare så började vi mäta stjärnor med rymdteleskop, det första då döpt till just Hipparcos. Detta efterföljdes av rymdteleskopet Gaia, som bidrar stort till forskningen bakom denna avhandling. Detta ger oss en väldigt noggran bild av hur Vintergatans kinematik ser ut idag. För att fullända bilden, använder man sig dessutom utav numeriska simulering som kan återskapa de resultat som vi ser i mätdatan. Föreningen av teori och observationer förstärker då varandra. 

% Den första artikeln i denna avhandling använder just numeriska simuleringar för att titta på hur stjärnor förflyttar sig in och ut från Galaxen på grund av interaktioner med dess spiralarmar och centrala stav. Mer specifikt hur denna migrering beror på Galaxskivans styrka och hur vertikalt utsträckt sjärnans omloppsbana går. Med över 100 simulerade skivor kunde vi bestämma att i gravitationellt svaga skivor migrerar mest stjärnorna nära skivan och i starka skivor migrerar de oavsett vertikal omloppsbana. 

% I den andra och tredje artikeln använde vi mätdatan från Gaia. Genom att vi använder positioner och hastigheter tangentiellt på himlen, utan att kräva hastigheter i siktriktningen, hade vi tillgång till extremt stora mängder data. Vi kan då lokalt få en uppskattning av stjärnornas hastighetsfördelning, även fast vi saknar en hastighet. Detta gjorde vi för tre olika urval av data. I artikel två använde vi vita dvärgar, vilket är kvarlevorna av döda stjärnor, och kunde upptäcka att där fanns två separata kinematiska populationer. I artikel tre använde vi Solens kvarter av stjärnor i skivan och en lokal del av Galaxens halo. Där lyckas vi identifiera många kända strukturer i hastighetsfördelningen, samt några nya som då tillhör den ansamlade halon, och skulle kunna vara kvarlevorna av ansamlade dvärggalaxer. 


