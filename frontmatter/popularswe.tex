\sectionhidenum{Populärvetenskaplig sammanfattning}

%Att utforska världen vi lever i är en del av människans nyfikna natur. Inom astronomi använder vi denna nyfikenhet till att observera universums stjärnor för att försöka förstå hur allt blev till och hur allt hänger samman. I arbetet som presenteras i denna avhandling har vi observerat stjärnor belägna i centrum av vår galax, Vintergatan. Från studier gjorda på andra galaxer vet vi att centrum av en galax hänger samman med hur hela galaxen ser ut och beter sig. Detta samband pekar på att mittpunkten och galaxen som helhet troligtvis utvecklats parallellt.

%Då Vintergatan och dess centrum är den mest närbelägna galaxen vi har, kan vi studera dess egenskaper i stor detalj, jämfört med andra galaxcentra. Om vi antar att Vintergatan är varken mer eller mindre speciell än andra galaxer, kan en god förståelse för Vintergatan och dess centrum lägga grunden för en större förståelse för hur galaxer i allmänhet hänger samman med sina centra.

%I vårt arbete hittar vi både likheter och skillnader mellan stjärnorna som finns i Vintergatans centrum respektive längre ut. Stjärnor består främst av väte och helium men kan även vara berikade med andra grundämnen, beroende på hur och när de har bildats. Både i centrum och längre ut i galaxen hittar vi stjärnor som haltmässigt ser ut att vara lika. Men när det kommer till de mest berikade stjärnorna finns det tydliga skillnader, där centrumstjärnorna har en högre halt kisel jämfört med stjärnor längre ut. Detta kan vara en indikation på att dessa regioner har bildats och utvecklats på olika sätt. Denna kunskap, om man lyckats bekräfta resultaten i framtida studier, är en viktig byggsten i vår förståelse kring Vintergatans utveckling. Som en förlängning kan även denna kunskap användas på andra galaxer och förhoppningsvis ge viktiga ledtrådar till galaxbildning och galaxutveckling i allmänhet.
