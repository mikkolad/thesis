\sectionhidenum{Populärvetenskaplig sammanfattning}

Utav alla ting på himlavalvet så är det största och mest slående det diffusa band av stjärnor som uppgör Vintergatan, vår hemgalax. För att förstå hur den skapades och utvecklas behöver vi, bland annat, förstå i detalj hur stjärnorna rör sig och var det befinner sig, ett forskningsfält som kallas astrometri. Den första stjärnkatalogen kom 200 f.v.t. från Hipparkos i antika Grekland. Två millennier senare så började vi mäta stjärnor med rymdteleskop, det första då döpt till just Hipparcos. Detta efterföljdes av rymdteleskopet \textit{Gaia} som sattes i omloppsbana 2013 och har revolutionerat astrometrin genom att bidra med en fantastisk stjärnkatalog av positioner och hastigheter. Denna katalog har bidragit stort till forskningen bakom denna avhandling. \textit{Gaia} ger oss en väldigt noggran bild av hur Vintergatans kinematik ser ut idag. För att fullända bilden, använder man sig dessutom utav numeriska simuleringar som kan återskapa och förklara de resultat som vi ser i mätdatan. Föreningen av teori och observationer förstärker då varandra och är kritisk för vår förståelse av Vintergatan. 

Den första artikeln i denna avhandling använder just numeriska simuleringar för att studera hur interaktioner med Galaxens spiral armar och centrala stav förflyttar stjärnor radiellt i skivans plan. Vi visar att denna migrering beror på Galaxskivans styrka och hur vertikalt utsträckt stjärnans omloppsbana är. Med över 100 simulerade skivor kunde vi bestämma att i mindre massiva skivor migrerar stjärnor mest nära skivan och i motsatt fall med en massiv skiva migrerar de oavsett hur långt utanför skivan som omloppsbanan går. 

I den andra och tredje artikeln använder vi mätdatan från \textit{Gaia}. Genom att vi använder positioner och hastigheter längst med himlavalvet, utan att kräva hastigheter i siktriktningen, får vi tillgång till extremt stora mängder data. Vi kan då uppskatta stjärnornas lokala hastighetsfördelning, trots att vi saknar en hastighet. Detta gjorde vi för tre olika urval av data. I den andra artikeln använder vi vita dvärgar, som är kvarlevorna av mindre massiva döda stjärnor, och kunde upptäcka att där fanns två separata kinematiska populationer. I tredje artikeln använder vi Solens kvarter av stjärnor i skivan och en lokal del av Galaxens halo. Där lyckas vi identifiera många kända strukturer i hastighetsfördelningen, samt några nya som tillhör den ansamlade halon, och skulle kunna vara kvarlevorna av uppslukade dvärggalaxer. 

% The first article in this thesis uses numerical simulations to study how interactions with the Galaxy's spiral arms and bar transports stars radially in the plane of the disc. We find that this migration depends on the Galactic disc's strength and how vertical extended the stellar orbit is. With over 100 simulated discs we could determine that in less massive discs mostly the stars close to the disc migrate and in the opposite case of massive discs they migrate regardless of how far above the disc the orbit goes.

% In the second and third articles we use data from \textit{Gaia}. By using positions and velocities along the celestial sphere, without needing velocities in the line-of-sight direction, we have access to extremely large amounts of data. This way, we obtain an estimate for the local stellar velocity distribution, despite lacking one velocity component. This was done for three samples of data. In the second article we used white dwarfs, which are the remains of low mass dead stars, and could discover that there were two separate kinematic populations. In the third article we used the Solar neighbourhood of stars in the disc and a local part of the Galaxy's halo. We were able to identify many known structures in the velocity distribution, as well as some new ones which then belong to the accreted halo, and could be the remains of accreted dwarf galaxies. 
