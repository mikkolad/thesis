\sectionhidenum{Populärvetenskaplig sammanfattning}

Utav alla ting på himlavalvet så är det största och mest slående det diffusa band av stjärnor som upppgör Vintergatan, vår hemgalax. För att förstå hur den skapades och utvecklas behöver vi, bland annat, förstå i detalj hur stjärnorna rör sig och var det befinner sig. Den första stjärnkatalogen kom 200 f.kr från Hipparchos i antika Grekland. Två millenier senare så började vi mäta stjärnor med rymdteleskop, det första då döpt till just Hipparcos. Detta efterföljdes av rymdteleskopet Gaia, som bidrar stort till forskningen bakom denna avhandling. Detta ger oss en väldigt noggran bild av hur Vintergatans kinematik ser ut idag. För att fullända bilden, använder man sig dessutom utam numeriska simulering som kan återskapa de resultat som vi ser i mätdatan. Föreningen av teori och observationer förstärker då varandra. 

Den första artikeln i denna avhandling använder just numeriska simuleringar för att titta på hur stjärnor förflyttar sig in och ut från Galaxen på grund av interaktioner med dess spiralarmar och centrala stav. Mer specifikt hur denna migrering beror på Galaxskivans styrka och hur vertikalt utsträckt sjärnans omloppsbana går. Med över 100 simulerade skivor kunde vi bestämma att i gravitationellt svaga skivor migrerar mest stjärnorna nära skivan och i starka skivor migrerar de oavsett vertikal omloppsbana. 

I den andra och tredje artikeln använde vi mätdatan från Gaia. Genom att vi använder positioner och hastigheter tangentiellt på himlen, utan att kräva hastigheter i siktriktningen, hade vi tillgång till extremt stora mängder data. Vi kan då lokalt få en uppskattning av stjärnornas hastighetsfördelning, även fast vi saknar en hastighet. Detta gjorde vi får tre olika urval av data. I artikel två använda vi vita dvärgar, vilket är kvarlevorna av döda stjärnor, och kunde upptäcka att där fanns två separata kinematiska populationer. I artikel tre använde vi Solens kvarter av stjärnor i skivan och en lokal del av Galaxens halo. Där lyckas vi identifiera många kända strukturer i hastighetsfördelningen, samt några nya som då tillhör den ansamlade halon, och skulle kunna vara kvarlevorna av ansamlade dvärggalaxer. 
